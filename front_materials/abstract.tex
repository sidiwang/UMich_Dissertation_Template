In Duchenne muscular dystrophy (DMD) and other rare diseases, recruiting patients into clinical trials is challenging. Additionally, assigning patients to long-term, multi-year placebo arms raises ethical and trial retention concerns. This poses a significant challenge to the traditional sequential drug development paradigm. In this dissertation, we present snSMART, or small sample, sequential, multiple assignment, randomized trial, designs and methods that formally incorporate external control data under both the non-longitudinal and longitudinal settings. 

Following an introduction of an snSMART and external control data integration in Chapter I, in Chapter II, we propose an snSMART design that combines dose selection and confirmatory assessment into a single trial. This multi-stage design evaluates the effects of multiple doses of a promising drug and rerandomizes patients to appropriate dose levels based on their stage 1 dose and response. Our proposed approach increases the efficiency of treatment effect estimates by i) enriching the placebo arm with external control data, and ii) using data from all stages. Data from external control and different stages are combined using a robust Meta-Analytic Combined (MAC) approach to consider the various sources of heterogeneity and potential selection bias. We reanalyze data from a DMD trial using the proposed method and external control data from the Duchenne Natural History Study (DNHS). Our method's estimators show improved efficiency compared to the original trial. Also, the robust MAC-snSMART method most often provides more accurate estimators than the traditional analytic method. Overall, the proposed methodology provides a promising candidate for efficient drug development in DMD and other rare diseases.

In Chapter III, we plan to extend the notion introduced in Chapter II to clinical trials with longitudinal data. In DMD trials, it is common to assess drug efficacy at multiple time points. The snSMART design introduced in Chapter II can be easily extended to include longitudinal data in each stage. For the primary analysis, we use inverse probability treatment weighting (IPTW) to address the confounding variables and evaluate the treatment effect through the Bayesian mixed model for repeated measure (MMRM). Data from external control are combined using a robust Meta-Analytic Predictive (MAP) approach to handle possible prior-data conflicts. Simulation studies will be performed to assess the performance of our proposed method against traditional analytic method.

Finally + OSPRED 

