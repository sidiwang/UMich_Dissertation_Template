In \ac{DMD} and other rare diseases, recruiting patients into clinical trials is challenging. Additionally, assigning patients to long-term, multi-year placebo arms raises ethical and trial retention concerns. This poses a significant challenge to the traditional sequential drug development paradigm. In this dissertation, we present \ac{snSMART} designs and methods that formally incorporate external control data under both the non-longitudinal and longitudinal settings.

After introducing the integration of \ac{snSMART} with external control data in Chapter \ref{chpt:introduction}, Chapter \ref{chpt:snSMART} proposes an \ac{snSMART} design that integrates dose selection and confirmatory assessment into a single trial. This multi-stage design evaluates the effects of multiple doses of a promising drug, rerandomizing patients to appropriate dose levels based on their stage 1 dose response. Our approach enhances the efficiency of treatment effect estimates by: (i) enriching the placebo arm with external control data, and (ii) utilizing data from all stages. We combine data from external controls and different stages using a robust \ac{MAC} approach, accounting for various sources of heterogeneity and potential selection bias. Upon reanalyzing data from a \ac{DMD} trial with our proposed method, MAC-snSMART, we observe that MAC-snSMART estimators offer improved efficiency over the original trial results. The robust MAC-snSMART method frequently provides more accurate estimators than traditional analytical methods. Overall, our proposed methodology provides a promising candidate for efficient drug development in \ac{DMD} and other rare diseases.

In Chapter \ref{chpt:longitudinal}, we present \ac{BLPM}, a notable advancement on the robust MAC-snSMART method from Chapter \ref{chpt:snSMART}. This enhancement introduces significant improvements to \ac{snSMART} research by: (1) enabling longitudinal data analysis, (2) incorporating patient baseline characteristics, (3) utilizing multiple imputation for missing data, (4) reducing heterogeneity with \ac{PS}, and (5) managing stage-wise treatment effect non-exchangeability. These developments significantly increase the snSMART design's utility and efficiency in rare disease drug development. \ac{BLPM} applies \ac{PS} trimming, \ac{IPTW}, and the \ac{MAC} framework to navigate heterogeneity and cross-stage treatment effects. Our evaluations, through simulation studies and the reanalysis of a \ac{DMD} trial, show that \ac{BLPM} methods consistently achieve the lowest \ac{rMSE} across tested scenarios, underscoring its potential to enhance rare disease drug development.

In Chapter \ref{chpt:survival}, we propose a multivariate, joint modeling approach to assess the underlying dynamics of \ac{PFS} components to forecast the death times of trial participants. Through \ac{BMA}, our proposed method improves the accuracy of the \ac{OS} forecast by combining joint models developed from each granular component of \ac{PFS}. A case study of a renal cell carcinoma trial is conducted, and our method provides the most accurate predictions across all tested scenarios. The reliability of our proposed method is verified through extensive simulation studies, which include a scenario where \ac{OS} is completely independent of \ac{PFS}. Overall, the proposed methodology emerges as a promising candidate for reliable \ac{OS} prediction in solid tumor oncology studies.

