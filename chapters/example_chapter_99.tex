\chapter{Summary and Future Work}
\label{chpt:conclusion}
Motivated by two phase 2/3 \ac{DMD} trials, SPITFIRE (NCT03039686) and tadalafil (NCT01865084), this dissertation proposes innovative trial designs and Bayesian methods for rare diseases such as \ac{DMD} to formally incorporate external data. Additionally, inspired by a renal cell carcinoma trial, it introduces Bayesian methods for landmark survival time prediction in oncology trials. This work aims to provide promising alternatives to the current drug development paradigm in rare diseases and a robust method for predicting patient survival times in ongoing trials.

Chapter 2 presents an \ac{snSMART} design comparing two dose levels with a continuous outcome, incorporating external controls formally to reduce the number of subjects needed in the placebo arm. The proposed robust MAC-snSMART method yields the most accurate and robust estimators among all tested methods when the assumption of stage-wise treatment effect exchangeability is met. This model leverages summary statistics from both external controls and current trials, employing ``change from baseline" as the outcome measure. This design and method efficiently utilize ``all" available data sources, namely external controls and all stages of the current trial. It supports subject recruitment and retention and aligns with the goals of the \ac{FDA}'s Complex Innovative Design program.

Chapter 3 proposes a multivariate joint modeling approach that provides real-time \ac{OS} predictions based on joint models between different components of \ac{PFS} and \ac{OS}. This approach has proven to provide reliable and robust predictions in our case study and has been validated across various scenarios in simulation studies. Such predictive power is invaluable for planning oncology trials and managing end-of-life care for patients. It can also be readily adapted to other types of solid tumors beyond renal cell carcinoma. Furthermore, this chapter enhances the copula model and the multi-state model to predict survival times for patients currently undergoing treatment, an application not previously explored before this work was published. Our proposed innovative modeling approach holds significant promise for benefiting the drug development process in the realm of solid tumors.

Building on the trial design proposed in Chapter 2, we significantly expand the application area of \ac{snSMART} by introducing a Bayesian longitudinal piecewise meta-analytic combined model. This model not only dynamically incorporates external control data but also utilizes subject-level longitudinal outcomes and baseline characteristics to further enhance the power and efficiency of trial analysis. Moreover, the proposed \ac{BLPM} methods address both between-trial non-exchangeability and stage-wise treatment effect non-exchangeability. This significantly improves the model's ability to tackle real-life challenges practitioners encounter. We compared the \ac{BLPM} methods with the \ac{BJSM} proposed by \cite{fang2023comparing} and traditional MMRM methods through simulation studies and example data analysis. Among all tested scenarios, the \ac{BLPM} provides the most efficient estimators.

Regarding Chapters 2 and 4, future work may consider adapting the proposed models to handle various types of outcomes, such as binary or categorical outcomes. Additionally, the proposed \ac{snSMART} design could be expanded to include interim analyses that allow for early termination of the study. Given the enrichment of the placebo arm with external controls, developing a method to accurately calculate the \ac{ESS} of external controls is crucial for assisting in the determination of the sample size needed for the trial.

Regarding Chapter 3, future work may consider incorporating individual-level weights into the model to provide personalized predictions for each subject. The current model, implemented in \textit{JAGS} and run under the \textit{rjags} package, converges quite slowly. Future efforts could explore strategies to accelerate model convergence, thereby making its use more time efficient.

In conclusion, this dissertation introduces innovative Bayesian methods for rare disease clinical trial design and survival time prediction in solid tumor clinical trials. We hope that our proposed trial design will lead to more approved drugs for patients with rare diseases, inspire future innovative trial designs, assist in planning oncology clinical trials, and ultimately improve patient well-being.



