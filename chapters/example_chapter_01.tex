\chapter{Introduction}
\label{chpt:introduction}

\section{Background on snSMART}
Conducting randomized controlled clinical trials (RCTs), the most rigorous way of
estimating the efficacy or effectiveness of treatment, can be difficult when the number of individuals affected is small. While useful in small samples, traditional methods such as crossover and N-of-1 trials have limitations that restrict their use and often encounter problems with recruitment and retention. In order to address these challenges, a new approach known as snSMART (small sample, sequential, multiple assignment, randomized trial) has been developed in recent years.

The snSMART design is similar to a classic \ac{SMART} design in that it first randomizes participants to one of several first-stage treatments and then conducts a second stage randomization based on the outcome of the first stage. However, the snSMART design differs in that it measures the same treatment outcome at the end of both stages and the time length of both stages is equal. Additionally, the goal of an snSMART is to identify the superior first-stage treatment using data from both stages, as opposed to a SMART's goal of identifying an effective personalized two-stage treatment sequence.

The snSMART design is particularly useful for disorders or diseases that affect a small group of people and remain stable over the duration of the trial. Additionally, the snSMART design allows for the sharing of data across stages, resulting in more precise estimates of first-stage treatment effects. However, the snSMART design will have a smaller sample size and be less flexible than a classic SMART design, requiring a different set of analytic methods.

Recent years have seen significant progress in the development of statistical methods for analyzing trial data from different snSMART designs: an snSMART with three active treatments \citep{wei2018bayesian, wei2020sample, chao2020dynamic}, a group sequential snSMART with three active treatments \citep{chao2020bayesian}, an snSMART with placebo and two dose levels of one treatment \citep{fang2021bayesian}, and an snSMART with continuous outcomes \citep{hartman2021design}.

\section{Background on External Control Data Integration}
Integrating external control data in clinical trials is a topic of growing importance in the field of biostatistics. The ability to incorporate historical control data can improve the efficiency and power of clinical trials, allowing for smaller sample sizes and more precise estimation of treatment effects.

\cite{rosenbaum1983central} proposed using propensity score methods in observational studies to account for potential confounders. This approach has been extended to using historical control data in clinical trials \citep{viele2014use, pocock1976combination}. Propensity score methods are useful in adjusting for any observed covariates that may lead to bias in the treatment effect estimates.

\cite{ibrahim2000power} proposed using power prior distributions for regression models, which can be useful for incorporating historical control data. Power priors allow for the incorporation of historical information in a Bayesian framework, which can help to improve the precision of treatment effect estimates.

\cite{hobbs2011hierarchical} developed hierarchical commensurate and power prior models for adaptive incorporation of historical information in clinical trials. This approach can be useful for trials where the historical data may not be directly comparable to the current trial.

\cite{spiegelhalter2004bayesian} discussed Bayesian approaches to clinical trials and health-care evaluation, which emphasizes the use of prior information to improve the precision of treatment effect estimates. \cite{neuenschwander2010summarizing} proposed methods for summarizing historical information on controls in clinical trials. These methods can be used to extract relevant information from historical data to improve the power and precision of current trials.

\cite{schmidli2014robust} proposed robust meta-analytic-predictive (MAP) priors in clinical trials with historical control information. This approach combines historical data with current data in a meta-analytic framework to improve the precision of treatment effect estimates. \cite{neuenschwander2016use} discussed the use of co-data in clinical trials through meta-analytic combined (MAC). This approach utilizes both historical and concurrent control data to improve the precision of treatment effect estimates.

Overall, there is a growing body of literature on integrating external control data in clinical trials. The use of propensity scores, power prior distributions, hierarchical models and meta-analytic approach are promising approaches for incorporating historical control data in a statistically rigorous manner. With the growing focus on efficiency and precision in clinical trials, the use of historical control data is likely to become increasingly important in the field of biostatistics. Thus far, there are no formal methods to incorporate external control data in the analysis of an snSMART design.

\section{Summary of Objectives}
None of the existing snSMART designs take into account external control data, and there is no established method for integrating control data from multiple sources in a robust manner. In Chapter II, we aim to fill this gap by introducing a new snSMART design that incorporates external control data, allowing for a reduction in the number of participants on the placebo arm. In our proposed study design, eligible patients have a chance of 1:2:2 or 1:3:3 to be assigned to the placebo, low-dose or high-dose treatment group in the first stage of the trial. In the second stage, patients are either re-assigned or re-randomized to the same or a different dose depending on their initial treatment and the outcome of the first stage. For example, patients who received placebo in stage 1 will be re-randomized with an equal probability to either the low-dose or high-dose treatment group, regardless of their stage 1 response. Those who received the low-dose in stage 1 will continue on that dose if they responded positively or switch to the high-dose if they did not respond. Participants who first received high-dose and responded are re-randomized between low and high-dose, whereas those who did not respond to high-dose are taken off the trial in the second stage to discuss further treatment options with their physician. To utilize all external control data and information from both stages of the snSMART to enhance the accuracy of treatment effect estimations, we propose the MAC-snSMART method, which is a robust, flexible, hierarchical model to make inferences about stage 1. We simulated trials to compare results from the MAC-snSMART method to a traditional method where only stage 1 data is used for analysis. We compare the accuracy and efficacy of the treatment effect estimates. This work is currently under revision at \textit{Biometrics Practice}.

The next chapter shifts focus from clinical trial design to improving the prediction of overall survival in solid tumor oncology studies. In oncology clinical trials, the choice of endpoint is a complex process. Two commonly used endpoints are progression-free survival (PFS) and overall survival (OS). PFS is the length of time during and after treatment in which a patient lives without progression of disease (PD), while OS is the duration from the start of study treatment to the date of death due to any cause. OS remains the clinical gold standard for assessing patient benefit, however, powering a trial to show an OS benefit can be challenging. Factors such as longer duration of OS trials, patients crossing over to alternative treatment after progression, starting other anti-cancer therapy, or loss to follow-up can make it difficult. Additionally, OS data are often not mature enough to draw proper statistical inferences at the time of the primary analysis of PFS. Accurate prediction of OS can aid in resource allocation, future planning, and understanding the probability of success in oncology trials. It can also guide patient care and use of limited healthcare resources. In this project, we explore a model-based approach for forecasting the death times of trial participants using available, mature PFS data. We propose a multivariate joint modeling approach to assess the underlying dynamics of the PFS components (i.e., target lesion, non-target lesion, and new lesion) to predict OS. We build joint/marginal models based on OS and each component of the PFS, and obtain real-time OS predictions based on each model. In total, four groups of intermediate predictions are generated. The final OS prediction is derived based on all four models simultaneously using Bayesian model averaging (BMA) \citep{hoeting1999bayesian}. This topic has gained interest in the statistical and clinical literature and has implications for both drug development and patient care. We are currently finishing the work on this project.

In Chapter IV, we apply a similar trial design as in Chapter II, but with the addition of longitudinal data collected at each stage. Our goal remains the same- to estimate the stage 1 treatment effect efficiently by utilizing data from both stages of the trial and incorporating external data. This design is more versatile as it accommodates a larger amount of data from DMD natural history studies, which is collected over time. In order to evaluate the treatment effect in our proposed study design, we use a Bayesian mixed model for repeated measure (MMRM) and account for confounding variables through inverse probability treatment weighting (IPTW). MMRM is a statistical model that allows for the analysis of repeated measures data and accounts for the correlation between measurements taken on the same subject. IPTW is a statistical technique that helps to balance the distribution of covariates between the treatment groups, which in turn helps to reduce bias caused by confounding variables. Additionally, we incorporate external control data using a robust Meta-Analytic Predictive (MAP) approach to handle potential conflicts with prior data. This approach allows us to consider various sources of heterogeneity and potential selection bias when combining external control data and data from different stages of the trial. To evaluate the performance of our proposed method, we will conduct simulation studies comparing it to the traditional analytic method. This will provide insight into the strengths and limitations of our approach and help to identify the conditions under which it performs well. We have just started this project.